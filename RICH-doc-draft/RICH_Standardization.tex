\documentclass[]{article}

%opening
\title{RICH Standardization}
\author{Simon, Beshr}

\begin{document}

\maketitle

\begin{abstract}
Since the submission of the original RICH proposal, the IETF charted the 6TiSCH working group to develop “IPv6 over the TSCH mode of IEEE 802.15.4e" architectures, mechanisms, and standards. This encompasses the RICH technical goals (and re-confirms the relevance of RICH). In this catalyst the RICH team will monitor the developments in and contribute to the IETF 6TiSCH workgroup by * aligning the RICH Technology Stack and Demonstrator closer to the emerging 6TiSCH specifications where feasible, and * providing implementation questions and findings feedback to the 6TiSCH workgroup in the form of contributions to their mailing list.
\end{abstract}

\section{OBJECT}

\emph{What standard are you working on, and what are the technologies they could apply to? What's the current situation with the standards to which you want to contribute to?}

We are working on the IEEE802.15.4e standard – TSCH mode. This standard applies to low-power radios used with ZigBee technology. 
From a use case point of view, this standard is best suited for achieving reliable communication required in areas such as process automation.
The standard itself, however, has gaps which that could potentially make two standard-complaint implementations incompatible.
Moreover, the standard does not address at all how to operate in the 6LoWPAN stack, which is the standard for IPv6 operation in low power wireless personal area networks. 
For the aforementioned reasons, the community has spun off the IETF 6TiSCH workgroup which has produced a few Internet drafts.

\section{SCOPE \& INNOVATION DELTA}

\emph{What are the expected results of this task? What should be the advancement stage of the standard by the end of 2014, according to your plans?}

In this task we expect to deliver an implementation of RICH technology stack aligned to the emerging 6TiSCH standard (where feasible), which in itself an important contribution as it enables further experimental system research in this scope; especially that our project promises a testbed and open source software.

Besides, we believe that during the implementation process we will gain practical insights about the potential shortcomings and blurry aspects together with expected positive reinforcement of related Internet drafts. 
By the end of 2014, we will have an implementation featuring a representative realization of “IPv6 over the TSCH mode of IEEE 802.15.4e". 
To be more specific, we will address in a practical way the integration of RPL routing over TSCH together with a centralized scheduler, and the interfacing to IPv6 on the transport layer. 
We will provide feedback about these aspects and/or alternative solutions to those proposed in the related Internet drafts.

\section{ADDED VALUE}

\emph{How is this standard expected to boost the service/business proposition underpinning this Activity?}

This standard is of extreme importance since it regulates the integration with IPv6, without which otherwise will be meaningless to talk about interoperable operation among nodes from various suppliers.

\section{PLAN}

\emph{Which standardization bodies do you plan to address? What actions do you plan to carry out in order to achieve the objectives?}

IETF 6TiSCH workgroup is the body drafting the emerging standard. 
We will communicate our insights and discuss the potential shortcomings in the related mailing lists and forums.

\section{IMPACT}

\emph{What market(s) is this standard expected to foster? What business opportunities could thrive in such market(s)?}

This standard is expected to foster the market of building automation, home automation and city automation. 

This standard will enable businesses to offer low-cost wireless solutions and out-of-the-box services to 

\end{document}
